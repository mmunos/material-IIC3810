%!TEX root = main.tex

%--------------------------------------------------
\begin{frame}
\frametitle{La demostración de la proposición}

{\small

Nótese que $P^{n+m}[a,a] \geq P^{n}[a,a] \cdot P^{m}[a,a]$
\begin{itemize}
    \item Por lo tanto, tenemos la siguiente propiedad de clausura: si $n, m \in J(a)$, entonces $n + m \in J(a)$ 
\end{itemize}

\vs{6}

\visible<2->{
\begin{lema}
Para todo $a \in \Omega$, existe una par de números consecutivos en $J(a)$
\end{lema}
}

\vs{6}

\visible<3->{
{\bf Demostración:} suponga que existe $a \in \Omega$ tal que $J(a)$ no contiene algún par de números consecutivos
\begin{itemize}
    \item Hay una distancia mínima $k \geq 2$ entre los elementos de $J(a)$
\end{itemize}
}


}

\end{frame}

%--------------------------------------------------
\begin{frame}
\frametitle{La demostración de la proposición}

{\small

Tenemos entonces que existe $n_0 \in J(a)$ tal que $n_0+k \in J(a)$

\vs{8}

\visible<2->{
Además, existe $n_1 \in J(a)$ tal que $n_1 = m \cdot k + d$, con $m \in \mathbb{N}$ y $1 \leq d \leq k-1$
\begin{itemize}
    \item De otra forma, se tendría que $k|n$ para cada $n \in J(a)$, lo cual implicaría que $\text{MCD(}J(a)) \geq k > 1$
\end{itemize}
}

\vs{6}

\visible<3->{
Por lo tanto, por la propiedad de clausura: $(m+1) \cdot (n_0 + k) \in J(a)$ y $n_1 + (m+1) \cdot n_0 \in J(a)$
}

\vs{4}



}

\end{frame}

%--------------------------------------------------
\begin{frame}
\frametitle{La demostración de la proposición}

{\small

La distancia entre $(m+1) \cdot (n_0 + k)$ y $n_1 + (m+1) \cdot n_0$ es:
\begin{eqnarray*}
(m+1) \cdot (n_0 + k) - (n_1 + (m+1) \cdot n_0) &=& (m+1) \cdot k - n_1\\
&=& k - (n_1 - m \cdot k)\\
&=& k - d\\
& < & k
\end{eqnarray*}

\vs{6}

\visible<2->{
Encontramos entonces dos números en $J(a)$ cuya distancia es menor que~$k$, lo cual contradice el supuesto inicial.
\begin{itemize}
\item Concluimos que existe un par de números consecutivos en $J(a)$ \qed
\end{itemize} 
}

}

\end{frame}

%--------------------------------------------------
\begin{frame}
\frametitle{La demostración de la proposición}

{\small

\begin{lema}
Dado $k \in \mathbb{N}$ con $k \geq 1$, existe $n_0 \in \mathbb{N}$ tal que $n_0 \geq 1$ y:
\begin{eqnarray*}
\{ n \in \mathbb{N} \mid n \geq n_0\} & \subseteq & \{ n_1 \cdot k + n_2 \cdot (k + 1) \mid n_1, n_2 \in \mathbb{N} \text{ y } n_1 \geq 1 \}
\end{eqnarray*}
\end{lema}

\vs{8}

\visible<2->{

{\bf Demostración:} Dado $n \geq k^2$, existen $m \in \mathbb{N}$ y $d \in \{0, \ldots, k-1\}$ tales~que
\begin{eqnarray*}
n - k^2 &=& m \cdot k + d
\end{eqnarray*}

}

}

\end{frame}

%--------------------------------------------------
\begin{frame}
\frametitle{La demostración de la proposición}

{\small

Podemos entonces expresar $n$ como:
\begin{eqnarray*}
n &=& k^2 + m \cdot k + d\\
& = & (k - d + m)\cdot k + d \cdot (k + 1)
\end{eqnarray*}

\vs{6}

Dado que $k - d > 0$, concluimos que para $n_0 = k^2$ se tiene que $n_0 \geq 1$ y:
\begin{eqnarray*}
\{ n \in \mathbb{N} \mid n \geq n_0\} & \subseteq & \{ n_1 \cdot k + n_2 \cdot (k + 1) \mid n_1, n_2 \in \mathbb{N} \text{ y } n_1 \geq 1 \} 
\end{eqnarray*} \qed


}

\end{frame}

%--------------------------------------------------
\begin{frame}
\frametitle{La demostración de la proposición}

{\small

Fije $c \in \Omega$

\vs{6}

Por el primer lema, existe $k \in J(c)$ tal que $k + 1$ también es un elemento de $J(c)$
\begin{itemize}
\item Por la definición de $J(c)$ sabemos que $k \geq 1$
\end{itemize}

\vs{6}

Por lo tanto, por el segundo lema existe $n_c \geq 1$ tal que:
\begin{eqnarray*}
\{ n \in \mathbb{N} \mid n \geq n_c\} & \subseteq & \{ n_1 \cdot k + n_2 \cdot (k + 1) \mid n_1, n_2 \in \mathbb{N} \text{ y } n_1 \geq 1\} \\
&\subseteq& J(c)
\end{eqnarray*}
Nótese que la última inclusión es consecuencia de que $n + m \in J(c)$ si~$n,m \in J(c)$

}

\end{frame}

%--------------------------------------------------
\begin{frame}
\frametitle{La demostración de la proposición}

{\footnotesize

Sean $a,b \in \Omega$
\begin{itemize}
\item Como la cadena de Markov considerada es irreducible, existen $r_{a, c}, s_{c, b} \geq 1$ tales que $P^{r_{a,c}}[a, c] > 0$ y $P^{s_{c, b}}[c, b] > 0$
\end{itemize}

\vs{8}

\visible<2->{
Sea $n_{a,b} = r_{a, c} + n_c + s_{c,b}$
}

\vs{8}

\visible<3->{
Dado $n \geq n_{a,b}$ con $\ell = n - n_{a,b}$, tenemos que:
\begin{eqnarray*}
P^n[a,b] & = & P^{n_{a,b} + \ell}[a,b]\\
& = & P^{r_{a, c} + n_c + \ell + s_{c,b}}[a, b]\\
& \geq & P^{r_{a, c}}[a,c] \cdot P^{n_c + \ell}[c,c] \cdot P^{s_{c,b}}[c, b] \ > \ 0
\end{eqnarray*}
Nótese que $P^{n_c + \ell}[c,c] > 0$ dado que $\{ n \in \mathbb{N} \mid n \geq n_c\} \subseteq J(c)$
}

}

\end{frame}

%--------------------------------------------------
\begin{frame}
\frametitle{La demostración de la proposición}

{\small

\alert{Para todo $a,b \in \Omega$ y para todo $n \geq n_{a,b}$, tenemos que $P^n[a,b] > 0$}

\vs{8}

\visible<2->{
Sea ${\displaystyle n^\star  =  \max_{a,b \in \Omega} n_{a, b}}$
}

\vs{8}

\visible<3->{
Tenemos que $P^{n^\star}[a,b] > 0$ para todo $a,b \in \Omega$
\begin{itemize}
\item Concluimos que $P$ es cuasipositiva (dado que $P$ es no negativa) \qed
\end{itemize}
}

}

\end{frame}


%--------------------------------------------------
\begin{frame}
\frametitle{Existencia del límite: cuasipositividad}

{\footnotesize

Sea $\{ X_t \}_{t \in \mathbb{N}}$ una cadena de Markov con conjunto finito de estados $\Omega$ y matriz de transición $P$, donde $P$ es cuasipositiva.

\vs{4}

\begin{teorema}
Si $a \in \Omega$, se tiene que:
\vs{1}
\begin{enumerate}
\item ${\displaystyle \lim_{n \to \infty} \pr(X_n = a \mid X_0 = b)}$ existe para cada $b \in \Omega$

\vs{2}

\item ${\displaystyle \lim_{n \to \infty} \pr(X_n = a \mid X_0 = b) =\lim_{n \to \infty} \pr(X_n = a \mid X_0 = c)}$ para cada~$b, c \in \Omega$
\end{enumerate}
\end{teorema}

\vs{8}

\visible<2->{
Obtenemos entonces como corolario el teorema que queríamos demostrar, dada la proposición anterior.}


}

\end{frame}

%--------------------------------------------------
\begin{frame}
\frametitle{La demostración de la existencia del límite}

{\small

Fije $a \in \Omega$

\vs{8}

Dado $n \in \mathbb{N}$, definimos:
\alert{
\begin{eqnarray*}
m_a^{(n)} & = & \min_{b \in \Omega} P^n[a, b] \\
M_a^{(n)} & = & \max_{b \in \Omega} P^n[a, b]
\end{eqnarray*}
}

}
    
\end{frame}

%--------------------------------------------------
\begin{frame}
\frametitle{La demostración de la existencia del límite}

{\footnotesize

Tenemos que $m_a^{(n)} \leq m_a^{(n+1)}$, puesto que:
\begin{eqnarray*}
m_a^{(n+1)} & = & \min_{b \in \Omega} P^{n+1}[a, b] \\
& = & \min_{b \in \Omega} \sum_{c \in \Omega} P^n[a, c] \cdot P[c, b] \\
& \geq & \min_{b \in \Omega} \sum_{c \in \Omega} \bigg(\min_{d \in \Omega} P^n[a, d]\bigg) P[c, b] \\
& = & \min_{b \in \Omega} \bigg(\min_{d \in \Omega} P^n[a, d]\bigg) \sum_{c \in \Omega} P[c, b] \\
& = & \min_{b \in \Omega} \bigg(\min_{d \in \Omega} P^n[a, d]\bigg)\\
& = & \min_{d \in \Omega} P^n[a, d] \ = \ m_a^{(n)}
\end{eqnarray*}

\vs{4}

\visible<2->{
De la misma forma se puede demostrar que $M_a^{(n)} \geq M_a^{(n+1)}$
}

}

\end{frame}


%--------------------------------------------------
\begin{frame}
\frametitle{La demostración de la existencia del límite}

{\footnotesize

\begin{lema}
Existen $r_a, s_a \in \mathbb{R}$ tales que:
\begin{eqnarray*}
\lim_{n \to \infty} m_a^{(n)} & = & r_a\\
\lim_{n \to \infty} M_a^{(n)} & = & s_a
\end{eqnarray*}
\end{lema}

\vs{8}

\visible<2->{
\begin{ejercicio}
Demuestre el lema considerando que $(m_a^{(n)})_{n \in \mathbb{N}}$ es una sucesión de números reales monotona creciente y acotada superiormente.
\begin{itemize}
\item ¿Qué propiedades debe usar el caso de la sucesión $(M_a^{(n)})_{n \in \mathbb{N}}$?
\end{itemize}
\end{ejercicio}}

}

\end{frame}


%--------------------------------------------------
\begin{frame}
\frametitle{La demostración de la existencia del límite}

{\small

Para demostrar el teorema basta entonces demostrar que:
\begin{eqnarray*}
\alert{\lim_{n \to \infty} \big(M_a^{(n)} - m_a^{(n)}\big)} & \alert{=} & \alert{0}
\end{eqnarray*}

\vs{8}

De hecho, si esto se cumple, para cada $b \in \Omega$ se tiene que:
\begin{eqnarray*}
\lim_{n \to \infty} \pr(X_n = a \mid X_0 = b) & = & r_a
\end{eqnarray*}
Dado que \ $\pr(X_n = a \mid X_0 = b) = P^n[a,b]$,\ $m_a^{(n)} \leq P^n[a,b] \leq M_a^{(n)}$ y~~$r_a = s_a$

}

\end{frame}

%%--------------------------------------------------
%\begin{frame}
%\frametitle{Existencia de un límite: cuasipositividad}
%
%{\small
%
%A partir de lo anterior necesitamos demostrar que existe un número $L_a \in \mathbb{R}$ tal que para todo $\delta > 0$ existe un número $n_1 \in \mathbb{N}$ tal que para todo $n > n_1$
%
%\begin{eqnarray*}
%M_a^{(n)} - L_a < \delta
%\end{eqnarray*}
%
%\visible<2->{
%
%dado que de esta forma podemos concluir que
%
%\begin{eqnarray*}
%L_a & = & {\displaystyle \lim_{n \to \infty}} M_a^{(n)} \\
%& = & {\displaystyle \lim_{n \to \infty}} M_a^{(n)} - {\displaystyle \lim_{n \to \infty}} (M_a^{(n)} - m_a^{(n)}) \\
%& = & {\displaystyle \lim_{n \to \infty}} (M_a^{(n)} - M_a^{(n)} + m_a^{(n)}) \\
%& = & {\displaystyle \lim_{n \to \infty}} m_a^{(n)} \\
%& = & {\displaystyle \lim_{n \to \infty} \pr(X_n = a \mid X_0 = b)}
%\end{eqnarray*}
%
%}
%
%}
%
%\end{frame}
%
%%--------------------------------------------------
%\begin{frame}
%\frametitle{Existencia de un límite: cuasipositividad}
%
%{\small
%
%Dado un $n \in \mathbb{N}$ y $a \in \Omega$, tenemos:
%
%\begin{eqnarray*}
%M_a^{(1)} \geq M_a^{(2)} \geq \dots \geq M_a^{(n)} \geq m_a^{(n)} \geq \dots \geq m_a^{(1)}
%\end{eqnarray*}
%
%\vs{4}
%
%\visible<2->{
%
%Ahora, escojamos un $L_a \in \mathbb{R}$ tal que para todo $n \in \mathbb{N}$
%
%\begin{eqnarray*}
%M_a^{(n)} \geq L_a \geq m_a^{(n)}
%\end{eqnarray*}
%
%}
%
%\vs{4}
%
%\visible<3->{
%
%Además, para $\delta > 0$, tomemos $\varepsilon = \delta$ y $n_1 = n_0$. Así, para todo $n > n_1$ tenemos:
%
%\begin{eqnarray*}
%M_a^{(n)} - L_a \leq M_a^{(n)} - m_a^{(n)} < \varepsilon = \delta \qed
%\end{eqnarray*}
%
%}
%
%}
%
%\end{frame}

%--------------------------------------------------
\begin{frame}
\frametitle{La demostración de la existencia del límite}

{\small

Sea $\ell = |\Omega|$, y sea $n_0 \in \mathbb{N}$ tal que $n_0 \geq 1$ y $P^{n_0}[b,c] > 0$ para cada $b,c \in \Omega$
\begin{itemize}
\item Además, sea $t = {\displaystyle \min_{b,c \in \Omega} P^{n_0}[b,c]}$
\end{itemize}


\vs{8}

\visible<2->{
Dados estados $b_0, c_0 \in \Omega$, defina:
\begin{eqnarray*}
\Omega_1 & = & \{d \in \Omega \mid P^{n_0}[d, b_0] \geq P^{n_0}[d, c_0]\} \\
\Omega_2 & = & \Omega \smallsetminus \Omega_1
\end{eqnarray*}
}

}

\end{frame}

%--------------------------------------------------
\begin{frame}
\frametitle{La demostración de la existencia del límite}

{\footnotesize

Nótese que:
\begin{eqnarray*}
\sum_{d \in \Omega_1} P^{n_0}[d, b_0] + \sum_{d \in \Omega_2} P^{n_0}[d, b_0] & = & \sum_{d \in \Omega} P^{n_0}[d, b_0]\\
& = & 1
\end{eqnarray*}

\vs{6}

\visible<2->{
A partir de lo cual obtenemos:
\begin{eqnarray*}
\sum_{d \in \Omega_1} \big(P^{n_0}[d, b_0] - P^{n_0}[d, c_0]\big) & = & 1 - \sum_{d \in \Omega_2} P^{n_0}[d, b_0] - \sum_{c \in \Omega_1} P^{n_0}[d, c_0] \\
& \leq & 1 - \sum_{d \in \Omega_2} t - \sum_{d \in \Omega_1} t \\
& = & 1 - \ell \cdot t
\end{eqnarray*}

}



}

\end{frame}


%--------------------------------------------------
\begin{frame}
\frametitle{La demostración de la existencia del límite}

{\footnotesize

Dado $n \geq 0$, tenemos que:
\begin{eqnarray*}
P^{n_0 + n}[a, b_0] - P^{n_0 + n}[a, c_0] & = & \sum_{d \in \Omega} P^{n}[a, d] \cdot P^{n_0}[d, b_0] - \sum_{d \in \Omega} P^{n}[a,d] \cdot P^{n_0}[d, c_0]\\
& = & \sum_{d \in \Omega} P^{n}[a, d] \cdot \big(P^{n_0}[d, b_0] - P^{n_0}[d, c_0]\big) \\
& = & \sum_{d \in \Omega_1} P^{n}[a, d] \cdot \big(P^{n_0}[d, b_0] - P^{n_0}[d, c_0]\big) + \\
&& \hspace{41pt} \sum_{d \in \Omega_2} P^{n}[a, d] \cdot \big(P^{n_0}[d, b_0] - P^{n_0}[d, c_0]\big) \\
& \leq & \sum_{d \in \Omega_1} M_a^{(n)} \cdot \big(P^{n_0}[d, b_0] - P^{n_0}[d, c_0]\big) + \\
&& \hspace{41pt} \sum_{d \in \Omega_2} m_a^{(n)} \cdot \big(P^{n_0}[d, b_0] - P^{n_0}[d, c_0]\big)
\end{eqnarray*}

}

\end{frame}

%--------------------------------------------------
\begin{frame}
\frametitle{La demostración de la existencia del límite}
\
{\footnotesize

Dado que:
\begin{eqnarray*}
\sum_{d \in \Omega_1} P^{n_0}[d, b_0] + \sum_{d \in \Omega_2} P^{n_0}[d, b_0] \ = \ 1 \ = \ 
\sum_{d \in \Omega_1} P^{n_0}[d, c_0] + \sum_{d \in \Omega_2} P^{n_0}[d, c_0]
\end{eqnarray*}

\vs{6}

Tenemos que:
\begin{eqnarray*}
\sum_{d \in \Omega_2} \big(P^{n_0}[d, b_0] - P^{n_0}[d, c_0]\big) = - \sum_{d \in \Omega_1} \big(P^{n_0}[d, b_0] - P^{n_0}[d, c_0]\big)
\end{eqnarray*}

\vs{6}

\visible<2->{
Concluimos entonces que:
\begin{multline*}
\sum_{d \in \Omega_1} M_a^{(n)} \cdot \big(P^{n_0}[d, b_0] - P^{n_0}[d, c_0]\big) + \sum_{d \in \Omega_2} m_a^{(n)} \cdot \big(P^{n_0}[d, b_0] - P^{n_0}[d, c_0]\big) 
\ = \\
\sum_{d \in \Omega_1} M_a^{(n)} \cdot \big(P^{n_0}[d, b_0] - P^{n_0}[d, c_0]\big) - \sum_{d \in \Omega_1} m_a^{(n)} \cdot \big(P^{n_0}[d, b_0] - P^{n_0}[d, c_0]\big) 
\end{multline*}
}

}

\end{frame}


%--------------------------------------------------
\begin{frame}
\frametitle{La demostración de la existencia del límite}

{\footnotesize

Así, dado que ${\displaystyle \sum_{d \in \Omega_1} \big(P^{n_0}[d, b_0] - P^{n_0}[d, c_0]\big) \leq 1 - \ell \cdot t}$, obtenemos:
\begin{eqnarray*}
P^{n_0 + n}[a, b_0] - P^{n_0 + n}[a, c_0] & \leq & \sum_{d \in \Omega_1} M_a^{(n)} \cdot \big(P^{n_0}[d, b_0] - P^{n_0}[d, c_0]\big) + \\
&& \hspace{41pt} \sum_{d \in \Omega_2} m_a^{(n)} \cdot \big(P^{n_0}[d, b_0] - P^{n_0}[d, c_0]\big)\\
& = & \sum_{d \in \Omega_1} M_a^{(n)} \cdot \big(P^{n_0}[d, b_0] - P^{n_0}[d, c_0]\big) - \\
&& \hspace{41pt} \sum_{d \in \Omega_1} m_a^{(n)} \cdot \big(P^{n_0}[d, b_0] - P^{n_0}[d, c_0]\big)\\
& = & \big(M_a^{(n)} - m_a^{(n)}\big) \sum_{d \in \Omega_1} \big(P^{n_0}[d, b_0] - P^{n_0}[d, c_0]\big)\\
& \leq & \big(1 - \ell \cdot t\big) \cdot \big(M_a^{(n)} - m_a^{(n)}\big)
\end{eqnarray*}

}

\end{frame}



%--------------------------------------------------
\begin{frame}
\frametitle{La demostración de la existencia del límite}

{\small

En particular, escogemos $b_0, c_0$ tales que
\begin{eqnarray*}
M_a^{(n_0 + n)} & = & P^{n_0 + n}[a, b_0]\\
m_a^{(n_0 + n)} & = & P^{n_0 + n}[a, c_0]
\end{eqnarray*}

\vs{8}

\visible<2->{
De esta forma, obtenemos:
\begin{eqnarray*}
\alert{M_a^{(n_0 + n)} -  m_a^{(n_0 + n)}} & \alert{\leq} & \alert{\big(1 - \ell \cdot t\big) \cdot \big(M_a^{(n)} - m_a^{(n)}\big)}
\end{eqnarray*}
}

}

\end{frame}